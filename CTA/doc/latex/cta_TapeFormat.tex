\chapter{Tape Format}

\section{Overview}

CTA uses the same AUL file format as CASTOR\footnote{CASTOR used several file formats over time, but by 2013,
only the AUL format was in use.}. This format is based on
\href{http://webstore.ansi.org/RecordDetail.aspx?sku=INCITS+27-1987\%5bS2008\%5d}{ANSI INCITS 27-1987}
and is described in detail on the
\href{http://it-dep-fio-ds.web.cern.ch/it-dep-fio-ds/Documentation/tapedrive/labels.html}{Tape Labels, ANSI and IBM}
web page (last updated in 2008).

The AUL format has the following descriptors:
\begin{itemize}
\item Volume Label (VOL1)
\item Header Blocks: Headers (HDR1, HDR2) and User Header Labels (UHL1)
\item Trailer Blocks: User Trailer Labels (UTL1)\footnote{The  UHLs and UTLs are defined in ANSI X 3.27. The general
description of the ANSI fields was documented in IBM's z/OS documentation.}
\end{itemize}
Each of these descriptor labels is contained in an 80-byte tape block of ASCII text. Empty bytes are stored as spaces
(0x20). The label descriptor must begin with the 4-byte identifier. Labels are terminated by a file mark: Tape Mark (TM)
or End of File (EOF).

\begin{table}[h!]
\caption{AUL label format}
\begin{center}
\begin{tabular}{ |c|c|c|c|c|c|c|c|c|c|c| }
  \hline
  VOL1 &  \cellcolor{orange}HDR1 & \cellcolor{orange}HDR2 & \cellcolor{orange}UHL1 & \cellcolor{green} TM &
   \cellcolor{gray} DATA & \cellcolor{green} TM & \cellcolor{orange}EOF1 & \cellcolor{orange}EOF2 & 
   \cellcolor{orange}UTL1 &  \cellcolor{green} TM \\
  \hline
  \multicolumn{1}{c|}{} &\multicolumn{10}{c|}{one data file} \\
\end{tabular}
\end{center}
\end{table}
Volumes that have just been initialised contain no data records, just a single `header label group':
\begin{table}[h!]
\caption{AUL prelabeled tape with one HDR1}
\begin{center}
\begin{tabular}{ |c|c|c| }
  \hline
  VOL1 &  \cellcolor{orange}HDR1(PRELABEL)  & \cellcolor{green} TM \\
  \hline
\end{tabular}
\end{center}
\end{table}

\section{Volume Label (VOL$n$)}

The very first label record on a labelled volume is VOL1. If this label is incorrect, you will not advance at all.

The format is shown in Table~\ref{tab:volume_label}. Example for beginning of the tape:
\begin{samepage}
\begin{small}
\begin{verbatim}
00000000  56 4f 4c 31 56 35 32 30  30 31 20 20 20 20 20 20  |VOL1V52001      |
00000010  20 20 20 20 20 20 20 20  20 20 20 20 20 20 20 20  |                |
00000020  20 20 20 20 20 43 41 53  54 4f 52 20 20 20 20 20  |     CASTOR     |
00000030  20 20 20 20 20 20 20 20  20 20 20 20 20 20 20 20  |                |
00000040  20 20 20 20 20 20 20 20  20 20 20 20 20 20 20 33  |               3|
\end{verbatim}
\end{small}
\end{samepage}

\begin{table}[ht]
\caption{The structure of the volume label}
\label{tab:volume_label}
\begin{center}
\begin{tabularx}{\textwidth}{ |c|c|c|X| }
  \hline
  \multicolumn{4}{|c|}{VOL1} \\
  \hline
  Bytes & Length & Offset & \multicolumn{1}{c|}{Content} \\
  \hline \hline
  0-3 & 4 & 0x00 & Volume label indicator: the characters \textbf{VOL1} \\
  \hline
  4-9 & 6 & 0x04 & Volume serial number (VSN) (e.g., ``AB1234'') \\
  \hline
  10 & 1 & 0x0A & Accessibility (left as empty space) \\
  \hline
  11-23 & 13 & 0x0B & Reserved (spaces) \\
  \hline
  24-36 & 13 & 0x18 & Implementation identifier (left as empty spaces) \\
  \hline
  37-50 & 14 & 0x25 & Owner identifier (the string ``CASTOR'' or STAGESUPERUSER name, padded with spaces)\\
  \hline
  51-78 & 28 & 0x33 & Reserved (spaces) \\
  \hline
  79 & 1 & 0x4F & Label standard level (1, 3 and 4 are listed as valid in IBM's documentation. CASTOR uses ASCII `3') \\
  \hline
\end{tabularx}
\end{center}
\end{table}

\section{Header Label (HDR$n$)}

HDR1 and HDR2 are normally found together at the beginning of a dataset.

The format for HDR1 is shown in Table~\ref{tab:hdr1_eof1_labels} and the format for HDR2 is shown in
Table~\ref{tab:hdr2_eof2_labels}.
Example for the empty tape with PRELABEL and one HDR1:
\begin{small}
\begin{verbatim}
00000000  56 4f 4c 31 56 35 32 30  30 31 20 20 20 20 20 20  |VOL1V52001      |
00000010  20 20 20 20 20 20 20 20  20 20 20 20 20 20 20 20  |                |
00000020  20 20 20 20 20 72 6f 6f  74 20 20 20 20 20 20 20  |     root       |
00000030  20 20 20 20 20 20 20 20  20 20 20 20 20 20 20 20  |                |
00000040  20 20 20 20 20 20 20 20  20 20 20 20 20 20 20 33  |               3|
00000050  48 44 52 31 50 52 45 4c  41 42 45 4c 20 20 20 20  |HDR1PRELABEL    |
00000060  20 20 20 20 20 56 35 32  30 30 31 30 30 30 31 30  |     V5200100010|
00000070  30 30 31 30 30 30 31 30  30 30 31 33 32 33 34 30  |0010001000132340|
00000080  31 33 32 33 34 20 30 30  30 30 30 30 43 41 53 54  |13234 000000CAST|
00000090  4f 52 20 32 2e 31 2e 31  33 20 20 20 20 20 20 20  |OR 2.1.13       |
\end{verbatim}
\end{small}
Example of HDR1 for the second file on the tape:
\begin{small}
\begin{verbatim}
00000000  48 44 52 31 31 32 41 31  36 30 43 33 38 20 20 20  |HDR112A160C38   |
00000010  20 20 20 20 20 56 35 32  30 30 31 30 30 30 31 30  |     V5200100010|
00000020  30 30 32 30 30 30 31 30  30 30 31 32 30 34 31 30  |0020001000120410|
00000030  31 32 30 34 31 20 30 30  30 30 30 30 43 41 53 54  |12041 000000CAST|
00000040  4f 52 20 32 2e 31 2e 31  32 20 20 20 20 20 20 20  |OR 2.1.12       |
\end{verbatim}
\end{small}
Example of HDR2 for the first file on the tape:
\begin{small}
\begin{verbatim}
00000000  48 44 52 32 46 30 30 30  30 30 30 30 30 30 30 20  |HDR2F0000000000 |
00000010  20 20 20 20 20 20 20 20  20 20 20 20 20 20 20 20  |                |
00000010  20 20 20 20 20 20 20 20  20 20 20 20 20 20 20 20  |                |
00000030  20 20 30 30 20 20 20 20  20 20 20 20 20 20 20 20  |  00            |
00000040  20 20 20 20 20 20 20 20  20 20 20 20 20 20 20 20  |                |
\end{verbatim}
\end{small}

\begin{table}[ht]
\caption{The structure of the HDR1, EOF1 labels}
\label{tab:hdr1_eof1_labels}
\begin{center}
\begin{tabularx}{\textwidth}{ |c|c|c|X| }
  \hline
  \multicolumn{4}{|c|}{HDR1, EOF1} \\
  \hline
  Bytes & Length & Offset & \multicolumn{1}{c|}{Content} \\
  \hline \hline
  0-3 & 4 & 0x00 & Header label: the characters ``HDR1 or EOF1'' \\
  \hline
  4-20 & 17 & 0x04 & File identifier: hexadecimal CASTOR NS file ID. 
  \texttt{nsgetpath -x} can be used to find the CASTOR full path name. Aligned to left.
  In case of prelabeled tape `PRELABEL' is used instead of file ID.\\
  \hline
  21-26 & 6 & 0x15 & The volume serial number of the tape.\\
  \hline
  27-30 & 4 & 0x1B & File section number:  a number (0001 to 9999) that indicates
  the order of the volume within the multivolume aggregate.
  This number is always 0001 for a single volume data set. \\
  \hline
  31-34 & 4 & 0x1F & File sequence number: a number that indicates
  the relative position of the data set within a multiple data set group (aggregate).
  CASTOR uses modulus for fseq by 10000 \\
  \hline
  35-38 & 4 & 0x23 & Generation number:  `0001' in CASTOR. \\
  \hline
  39-40 & 2 & 0x27 & Version number of generation: `00' in CASTOR. 
   \\
  \hline
  41-46 & 6 & 0x29 & Creation date:  Date when allocation begins for creating the
   data set. The date format is  cyyddd, where:
   c = century (blank=19; 0=20; 1=21; etc.)
   yy = year (00-99)
  ddd = day (001-366) \\
  \hline
  47-52 & 6 & 0x2F & Expiration date: year and day of the year when the data set may be 
  scratched or overwritten. The data is shown in the format cyyddd.
  It is always advisable to set the expiration date when a volume is being initialised 
  (`prelabelled') to be a date before the current date, so that writing to the tape 
  is immediately possible.  \\
  \hline
  53 & 1 & 0x35 & Accessibility: a code indicating the security status of the data set and
  `space' means no data set access protection.  \\
  \hline
  54-60 & 6 & 0x36 & Block count: This field in the trailer label shows the number of data
  blocks in the data set on the current volume. This field in the header label is always `000000'. \\
  \hline
  60-72 & 13 & 0x3C & System code of creating system: a unique code that identifies the system.
  CASTOR with CASTOR BASEVERSION number string. \\
  \hline
  73-79 & 7 & 0x49 & Reserved \\
  \hline
\end{tabularx}
\end{center}
\end{table}

\begin{table}[ht]
\caption{The structure of the HDR2, EOF2 labels}
\label{tab:hdr2_eof2_labels}
\begin{center}
\begin{tabularx}{\textwidth}{ |c|c|c|X| }
  \hline
  \multicolumn{4}{|c|}{HDR2, EOF2} \\
  \hline
  Bytes & Length & Offset & \multicolumn{1}{c|}{Content} \\
  \hline \hline
  0-3 & 4 & 0x00 & Header label: the characters ``HDR2 or EOF2'' \\
  \hline
  4 & 1 & 0x04 & Record format. An alphabetic character that indicates the
  format of the records in the associated data set. For the AUL it could be only: F - fixed length 
  (U - was used for HDR2 for prelabeled tapes)\\
  \hline
  5-9 & 5 & 0x05 & Block length in bytes (maximum).  For the block size  greater than 100000 the value is 00000. \\
  \hline
  10-14 & 5 & 0x0A & Record length in bytes (maximum). For the record size greater than 100000 the value is 00000. \\
  \hline
  15 & 1 & 0x0F & Tape density. Depends on the tape density values are following: `2' for D800, `3' for D1600, `4' for D6250  \\
  \hline
  16-33 & 18 & 0x10 & Reserved \\
  \hline
  34 & 2 & 0x22 & Tape recording technique. The only technique available for 9-track tape is odd parity with no translation.
  For a magnetic tape subsystem with Improved Data Recording Capability, the values are: `P '- Record data in compacted format, 
`  ' - Record data in standard uncompacted format. For CASTOR is is `P' if the drive configured to use compression (i.e. xxxGC) \\
  \hline
  35-49 & 14 & 0x24 & Reserved \\
  \hline
  50-51 & 2 & 0x32 & Buffer offset `00' for AL and AUL tapes \\
  \hline
  52-79 & 28 & 0x34 & Reserved \\
  \hline
\end{tabularx}
\end{center}
\end{table}

\section{User Header Label (UHL$n$)}

The format for UHL1 is shown in Table~\ref{tab:uhl1_utl1_labels}. Example for the second file on the tape:
\begin{small}
\begin{verbatim}
00000000  55 48 4c 31 30 30 30 30  30 30 30 30 30 32 30 30  |UHL1000000000200|
00000010  30 30 32 36 32 31 34 34  30 30 30 30 32 36 32 31  |0026214400002621|
00000020  34 34 43 45 52 4e 20 20  20 20 4c 58 43 32 44 45  |44CERN    LXC2DE|
00000030  56 35 44 32 53 54 4b 20  20 20 20 20 54 31 30 30  |V5D2STK     T100|
00000040  30 30 42 20 58 59 5a 5a  59 5f 42 31 20 20 20 20  |00B XYZZY_B1    |
\end{verbatim}
\end{small}

\begin{table}[h!]
\caption{The structure of the UHL1, UTL1 labels}
\label{tab:uhl1_utl1_labels}
\begin{center}
\begin{tabularx}{\textwidth}{ |c|c|c|X| }
  \hline
  \multicolumn{4}{|c|}{UHL1, UTL1} \\
  \hline
  Bytes & Length & Offset & \multicolumn{1}{c|}{Content} \\
  \hline \hline
  0-3 & 4 & 0x00 & User header label: the characters ``UHL1 or UTL1''. \\
  \hline
  4-13 & 10 & 0x04 & Actual file sequence number ( `0' padded from left ). \\
  \hline
  14-23 & 10 & 0x0E & Actual block size ( `0' padded from left ). \\
  \hline
  24-33 & 10 & 0x18 & Actual record length ( `0' padded from left ). \\
  \hline
  34-41 & 8 & 0x22 & Site : a part of the domain name uppercase.\\
  \hline
  42-51 & 10 & 0x2A & Tape mover host name uppercase without domain name. \\
  \hline
  52-59 & 8 & 0x34 & Drive manufacturer. \\
  \hline
  60-67 & 8 & 0x3C & Drive model (first 8 bytes from the field PRODUCT IDENTIFICATION in the SCSI INQUIRY replay). \\
  \hline
  68-79 & 12 & 0x44 & Drive serial number. \\
  \hline
\end{tabularx}
\end{center}
\end{table}

\section{Data Records}

After a `header label group', data records follow of any length and in any number. Eventually, an EOF will appear and
then a `trailer label group' is expected.

The data block size is configurable but in practice a block size of 256 KiB has been used everywhere.

\section{End of File (EOF$n$)}

EOF1 and EOF2 are normally found together at the end of a dataset.

Note that an End of Volume (EOV$n$) label will appear instead of EOF$n$ if this is the final label group on the volume,
but the dataset continues on another volume. EOV1 and EOV2 are only expected together and at the end of a volume.

\section{User Trailer Label (UTL$n$)}

The format for UTL1 is the same as UHL1 (Table~\ref{tab:uhl1_utl1_labels}).

\section{Checksums}

When a file is written to tape, an \href{http://www.zlib.net/manual.html#Checksum}{Adler32} checksum is computed on the
file. The main advantages of Adler32 are that it is faster to compute than CRC32 or MD5, and it is distributive when
computing the checksum for a multi-block file. This checksum is not stored on the tape; it is stored as metadata in the
Catalogue.

\textbf{Note:} The tape drives also compute a CRC32 checksum on each block, which is checked in firmware. This checksum
is not seen by the software.

