\chapter{CTA Basic Concepts}

\gls{cta} is operated by authorized administrators (AdminUsers) who issue CTA commands from authorized machines
(AdminHosts), using the CTA command line interface. All administrative metadata (such as tape, tape pools, storage
classes, etc.) is tagged with a {\tt creationLog} and a {\tt lastModificationLog} which say who\slash when\slash where
created them and last modified them. An administrator may create ({\tt add}), delete ({\tt rm}), change ({\tt ch}) or
list ({\tt ls}) any of the administrative metadata.

\glspl{tapepool} are logical groupings of tapes that are used by operators to separate data belonging to different
\glspl{vo}. They are also used to categorize types of data and to separate multiple copies of files so that they end up
in different buildings. Each tape belongs to one and only one tape pool.

\glspl{logicallibrary} are the concept that is used to link tapes and drives together. We use logical libraries to
specify which tapes are mountable into which drives, and normally this mountability criteria is based on location,
that is the tape has to be in the same physical library as the drive, and on read\slash write compatibility. Each tape
and each drive has one and only one logical library.

A \gls{storageclass} is assigned to each archive file to specify how many tape copies the file is expected to have.

\glspl{archiveroute} link storage classes to tape pools. An archive route specifies onto which set of tapes the
copies will be written. There is an archive route for each copy in each storage class, and normally there should
be a single archive route per tape pool.

So to summarize, an archive file has a storage class that determines how many copies on tape that file should
have. A storage class has an archive route per tape copy to specify into which tape pool each copy goes. Each tape
tool is made of a disjoint set of tapes. And tapes can be mounted on drives that are in their same logical library.

\section{Archiving a file with CTA}

CTA has a \hyperref[ctaeoscli]{CLI for archiving and retrieving files to\slash from tape}, that is meant to be used by an
external disk-based storage system with an archiving workflow engine such as \gls{eos}. A non-administrative \gls{user} in CTA is an EOS user
which triggers the need for archiving or retrieving a file to/from tape. A User normally belongs to a specific CTA
\gls{mountgroup} which specifies the \gls{mountpolicy}.

Here we offer a simplified description of the archive process:

\begin{enumerate}
\item EOS issues an \hyperref[ctaeosarchive]{archive command} for a specific file, providing its source path, its
      \gls{storageclass} and the \gls{user} requesting the archival.\label{archiveone}
\item CTA returns immediately an {\tt ArchiveFileID} which is used by CTA to uniquely identify files archived on tape.
      This ID will be kept by EOS for any operations on this file (such as retrieval).
\item Asynchronosly, CTA carries out the archival of the file to tape, in the following steps:
\begin{enumerate}
   \item CTA looks up the Storage Class provided by EOS and makes sure it has correct \glspl{archiveroute} to one or more
         \glspl{tapepool} (more than one when multiple copies are required by the Storage Class).
   \item CTA queues the corresponding archive job(s) to the proper Tape Pool(s).\label{archiveb}
   \item in the meantime each free tape drive queries the central ``scheduler'' for work to be done, by communicating
         its name and its \gls{logicallibrary}\label{archivec}.
   \item for each work request, CTA checks whether there is a free tape in the required Tape Pool (as specified
         in~\ref{archiveb}), that belongs to the desired Logical Library (specified in~\ref{archivec}).
   \item if that is the case, CTA checks whether the work queued for that Tape Pool is worth a mount, i.e. if it meets
         the archive criteria specified in the \gls{mountgroup} to which the User (specified in~\ref{archiveone}) belongs.
   \item if that is the case, the tape is mounted in the drive and the file gets written from the source path
         (specified in~\ref{archiveone}) to the tape.
   \item after a successful archival, CTA notifies EOS through an asynchronous callback.
\end{enumerate}
\end{enumerate}

An archival process can be canceled at any moment (even after correct archival, but in this case it's a {\tt delete})
through the {\tt delete archive} command.

\section{Retrieving a file with CTA}

Here we offer a simplified description of the retrieve process:

\begin{enumerate}
\item EOS issues a \hyperref[ctaeosretrieve]{retrieve command} for a specific file, providing its {\tt ArchiveFileID},
      desired destination path and the \gls{user} requesting the retrieval\label{retrieveone}.
\item CTA returns immediately.
\item Asynchronosly, CTA carries out the retrieval of the file from tape, in the following steps:
\begin{enumerate}
   \item CTA queues the corresponding retrieve job(s) to the proper tape(s) (depending on where the tape copies are
         located).\label{retrievea}
   \item in the meantime each free tape drive queries the central ``scheduler'' for work to be done, by communicating its
         name and its \gls{logicallibrary}.\label{retrieveb}
   \item for each work request CTA checks whether the Logical Library (specified in~\ref{retrieveb}) is the same of
         (one of) the tape(s) (specified in~\ref{retrievea}).
   \item if that is the case, CTA checks whether the work queued for that tape is worth the mount, i.e. if it meets the
         retrieve criteria specified in the \gls{mountgroup} to which the User (specified in~\ref{retrieveone}) belongs
   \item if that is the case, the tape is mounted in the drive and the file gets read from tape to the destination
         (specified in~\ref{retrieveone}).
   \item after a successful retrieval CTA notifies EOS through an asynchronous callback.
\end{enumerate}
\end{enumerate}

A retrieval process can be canceled at any moment prior to correct retrieval through the ``cancel retrieve'' command.

