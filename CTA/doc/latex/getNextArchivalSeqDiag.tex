\documentclass[10pt,a4paper]{article}
\setlength{\parindent}{0pt}
\setlength{\parskip}{1em}
\usepackage[margin=12mm]{geometry}
\usepackage{hyperref}

\usepackage[underline=true,rounded corners=false]{pgf-umlsd}

\begin{document}

\section{Getting the next archival mount and its file transfer jobs}

Figure \ref{fig:get_next} is a sequence diagram showing how a tape server gets the next archival mount from the scheduler and then the transfer jobs for that mount.
\begin{figure}[h]
  \resizebox{\linewidth}{!}{
    \begin{sequencediagram}
      \newthread{t}{:TapeServer}
      \newinst[5]{s}{:Scheduler}
      \newinst[5]{c}{:Catalogue}
      \newinst{m}{:Mount}
      \newinst{j}{:Job}
      \begin{call}{t}{1. getNextMount(logicalLibrary, drive)}{s}{TapeMount}
      	\begin{call}{s}{2. getTapesForWriting(logicalLibrary)}{c}{list{\textless}TapeForWriting\textgreater}
    	\end{call}
        \begin{messcall}{s}{3. create}{m}
        \end{messcall}
      \end{call}
      \postlevel
      \begin{messcall}{t}{4. getMountType()}{m}
      \end{messcall}
      \begin{messcall}{t}{5. getVid()}{m}
      \end{messcall}
      \begin{call}{t}{6. getNextJob(logicalLibrary, drive)}{m}{Job}
        \begin{messcall}{m}{7. create}{j}
        \end{messcall}
      \end{call}
      \postlevel
      \begin{messcall}{t}{8. getTransferDetails}{j}
      \end{messcall}
      \begin{messcall}{t}{9. complete()}{j}
      \end{messcall}
    \end{sequencediagram}
  }
  \caption{Getting the next archival mount and its file transfer jobs}
  \label{fig:get_next}
\end{figure}
\end{document}
