\documentclass[10pt,a4paper]{article}

\setlength{\parindent}{0pt} \setlength{\parskip}{1em}

% Fix PGF unhappiness \def\Hnewline{\H^^J}

\usepackage[margin=12mm]{geometry} \usepackage{hyperref}

\usepackage[underline=true,rounded corners=false]{pgf-umlsd}

\begin{document}

\section{Queuing an archive request in CTA}

Figure \ref{fig:queue_archive} is a sequence diagram showing how an EOS instance, the CTA front-end, scheduler,
catalogue and object-store interact in order to queue an archive request.  The sequence diagram has the following
three prerequisites: \begin{itemize}
  \item A CTA administrator has entered the archive route(s) for the storage class of the file to be archived.
  \item A CTA administrator has entered the archive mount policy for the owner of the file.  \item A file has
  successfully been written to an EOS instance and the workflow engine of that instance is about to queue a request
  to archive the file in CTA.
\end{itemize} The front-end is responsible for receiving and authorising requests from CTA administrators and EOS
instances and passing them on to the Scheduler and Catalogue objects.  Step 1 of the figure \ref{fig:queue_archive}
shows the EOS instance initiating the scenario by requesting the CTA front-end to queue an archive request.
The CTA front-end is an XRootD plugin and relies on the underlying xrootd daemon to authenticate the EOS instance.
Step 2 shows the front-end plugin authorising the EOS instance. Step 3 shows the front end passing on the now
authorised queue request to the scheduler.  The scheduler interacts with two persistent stores, the catalogue
which is responsible for storing all of the critical metadata of the CTA system and the object-store (OStoreDB)
which is responsible persisting the archive and retrieve request queues of CTA.  The catalogue and OStoreDB do
not communicate directly with each other, the scheduler is responsible for pulling information from one store and
pushing it into the other.  Step 4 shows the scheduler calling the prepareForNewFile() method of the catalogue in
order to get all of the information required to queue the request.  The prepareForNewFile() method of the catalogue
has to provide 4 pieces of information.  It has to generate and provide a unique archive identifier for the file,
it has to use the storage class of the file to lookup and provide the destination tape pool(s), it has to use the
owner of the file to lookup and provide the owner's archive mount policy and finally it has to serialise all of
the remaining metadata of the file into a blob that can be used once the file has actually been written to tape in
order to record the full result of the archived file transfer into the catalogue.  The catalogue only records the
end result of files being written to tape.  The catalogue does not contain any information about individual files
that are "on their way" to tape.  The blob will be kept within the OStoreDB until the moment when the file has been
written to tape, at which point the blob will be given back to the catalogue for permanent storage.  \begin{figure}[h]
  \resizebox{\linewidth}{!}{
    \begin{sequencediagram}
      \newthread{e}{:EOS} \newinst[3]{f}{:FrontEnd} \newinst[3]{s}{:Scheduler} \newinst[8]{c}{:Catalogue}
      \newinst{o}{:OStoreDB} \begin{call}{e}{1. queueArchiveRequest()}{f}{archiveFileId}
        \begin{call}{f}{2. authoriseRequest()}{f}{true} \end{call} \postlevel
        \begin{call}{f}{3. queueArchiveRequest()}{s}{archiveFileId}
          \begin{call}{s}{4. prepareForNewFile(storageClass, archiveRequest)}{c}{\shortstack{
            archiveQueueCriteria =\\ \{archiveFileId, tapePools, mountPolicy, archiveFileBlob\}}} \postlevel \postlevel
          \end{call} \postlevel \begin{messcall}{s}{5. queueArchiveRequest(archiveQueueCriteria)}{o} \end{messcall}
        \end{call}
      \end{call}
    \end{sequencediagram}
  } \caption{Queuing an archive request} \label{fig:queue_archive}
\end{figure} \newpage \section{Queuing a retrieve request in CTA}

Figure \ref{fig:queue_retrieve} is a sequence diagram showing how an EOS instance, the CTA front-end, scheduler,
catalogue and object store interact in order to queue a retrieve request.  The sequence diagram has the following
three prerequisites: \begin{itemize}
  \item A file has been written to EOS, archived to tape and then had just its EOS disk copy deleted.  The file is
  still referenced in the EOS namespace and has one or more copies safely stored on tape.  \item A CTA administrator
  has entered a recall mount policy for a user who wishes to retrieve the file back from tape.  \item The EOS user
  wanting the file has just requested EOS to retrieve it from tape.
\end{itemize} Step 1 of figure \ref{fig:queue_retrieve} shows EOS initiating the scenario by requesting the
CTA front-end to queue a request to retrieve the file.  In steps 2 and 3 the front-end authorises the request
and passes it on to the scheduler.  At step 4 the scheduler calls the prepareToRetrieveFile() method on the
catalogue.  The catalogue must return everything it knows about the file that is pertinent to its retrieval.
The prepareToRetrieveFile() method must therefore return the locations of all of the file's tape copies and the
policy to be used to decide when to mount a tape for retrieval.  If there is more than one copy on tape, then it
will be the responsibility of the scheduling algorithm to decide which one is finally passed on to the tape servers.
The location of a tape copy shall include the following 4 pieces of information: \begin{itemize}
  \item The volume identifier of the tape.  \item The file sequence number of the tape file.  \item The block ID
  of the tape file.
\end{itemize} A mount policy contains many criteria; an example of such a criteria is the minimum amount of data
required to warrant a tape mount.  The catalogue does not communicate with the object store (OStoreDB) directly and
therefore at step 5 the scheduler passes the result of prepareToRetrieveFile() onto the OStoreDB.  \begin{figure}[h]
  \resizebox{\linewidth}{!}{
    \begin{sequencediagram}
      \newthread{e}{:EOS} \newinst[3]{f}{:FrontEnd} \newinst[3]{s}{:Scheduler} \newinst[7]{c}{:Catalogue}
      \newinst{o}{:OStoreDB} \begin{messcall}{e}{1. queueRetrieveRequest()}{f}
        \begin{call}{f}{2. authoriseRequest()}{f}{true} \end{call} \postlevel
        \begin{messcall}{f}{3. queueRetrieveRequest()}{s}
          \begin{call}{s}{4. prepareToRetrieveFile(archiveFileId, user)}{c}{tapeFiles, mountPolicy} \postlevel
          \end{call} \postlevel \begin{messcall}{s}{5. queueRetrieveRequest(tapeFiles, mountPolicy)}{o} \end{messcall}
        \end{messcall}
      \end{messcall}
    \end{sequencediagram}
  } \caption{Queuing a retrieve request} \label{fig:queue_retrieve}
\end{figure}

\end{document}
