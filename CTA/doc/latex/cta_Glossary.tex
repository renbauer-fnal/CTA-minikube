% Acronyms

\newacronym{cta}{CTA}{CERN Tape Archive}
\newacronym{castor}{CASTOR}{CERN Advanced STORage Manager}
\newacronym{hsm}{HSM}{Hierarchical Storage Management}
\newacronym{sss}{SSS}{Simple Shared Secret}

\setacronymstyle{long-short-desc}

\newacronym[description={Refers to CERN experiments (ALICE, ATLAS, CMS, LHCb) or other groups which require separate data
storage, such as the International Linear Collider (ILC)}]{vo}{VO}{Virtual Organisation}

\newacronym[description=Data and Storage Services group in the CERN IT department]{dss}{DSS}{Data and Storage Services}

% Glossary entries

\newglossaryentry{archiveroute}
{
  name={Archive Route},
  description={Specifies the set of tapes on which the copies of an archive file will be written, i.e. the relationship
    between \glspl{storageclass} and \glspl{tapepool}. There is an archive route for each copy in each storage class.
    Normally there should be a single archive route per tape pool}
}

\newglossaryentry{eos}
{
  name={EOS},
  description={A disk-based, low-latency storage service with a highly-scalable hierarchical namespace, using the XRoot
    protocol for data access possible. (The name \textbf{EOS} is not an acronym, it was inspired by the Greek goddess of
    the dawn, $E\omega\sigma$)}
}

\newglossaryentry{logicallibrary}
{
  name={Logical Library},
  plural={Logical Libraries},
  description={Specifies which tapes are mountable into which drives. Each tape and each drive belongs to exactly one
    logical library. The mountability criteria is based on physical location (the tape and the drive must be in the same
    physical tape library) and on read\slash write compatibility}
}

\newglossaryentry{mountgroup}
{
  name={Mount Group},
  description={Creates a link between \glspl{user} and \glspl{mountpolicy}}
}

\newglossaryentry{mountpolicy}
{
  name={Mount Policy},
  plural={Mount Policies},
  description={Specifies the mount criteria and limitations that trigger a tape mount}
}

\newglossaryentry{storageclass}
{
  name={Storage Class},
  plural={Storage Classes},
  description={Specifies how many tape copies an archive file is expected to have}
}

\newglossaryentry{tapepool}
{
  name={Tape Pool},
  description={A logical grouping of tapes. Each tape belongs to exactly one tape pool. Tape pools are used to keep
    data belonging to different \acrshortpl{vo} separate, categorise types of data and to separate multiple copies of files
    so that they are physically stored in different buildings}
}

\newglossaryentry{user}
{
  name={User},
  description={An EOS user which triggers the archiving\slash retrieving of a file to\slash from tape}
}

% Print the glossary

\printglossaries

% Reset the use of all acronyms

\glsresetall
