\documentclass{lhcb+cta}

\newcommand{\thisminutesdate}{8 August 2018}

\begin{document}

This meeting was to discuss data transfers between the LHCb pit and T0. Niko and Tommaso are are the team responsible for this, including any software development. For transfers between T0 and T1 our contacts are Christophe and Jo\"{e}l.

\present{Michael Davis}{Niko Neufeld, Tommaso Columbo}

\section*{Workflow and Use Cases}

Data sent from the LHCb pit to CASTOR is transferred using Dirac and the XRoot protocol. Dirac uses a pull workflow, unlike Rucio which is based on a push workflow.

Monte Carlo data is generated directly in the filters during periods of no data taking, so from our perspective it is exactly the same as raw data. The only difference is that the rate at which it is generated is a lot less.

Besides writing raw data, they have a second use case where they sometimes bring back data from tape for reconstruction.

When raw data is sent to CASTOR, LHCb actively checks (a) that the file has been safely archived to tape, and (b) the checksum, before deleting the file from the pit.

They have been toying with the idea of having a separate EOS instance for the raw data. If they decide to go this way, then the LHCb pit will appear like a T1 to us.

\section*{The Way Forward}

Our proposal is that LHCb will send raw data to EOSLHCB and not EOSCTALHCB. They are fine with this change, but would like us to provide an API so that they can confirm when the file is safely on tape. If they have to poll EOSLHCB to get this information that's fine.

Perhaps the ``file on tape'' state will be provided as an XRoot extended attribute? If so it would require an update to an as yet-unreleased version of XRoot. This is also OK from their side but will require some coordination with the Dirac developers.

We need to have the same conversation around the workflow between T0 and T1. I spoke to Jo\"{e}l and will schedule another meeting with him and Christophe after Julien is back.

When we are ready, we should go back to them with our API proposal and work out a timeline for setting up a test instance.

\end{document}